%% !TEX TS-program = Arara
%% arara: pdflatex: {shell: yes}
%% arara: pdflatex: {shell: yes}
%% arara: clean: { extensions: [ log, aux, nav, out, snm, vrb, toc ] }
 
\documentclass[a4paper,ngerman,12pt]{exam}
% \listfiles

\usepackage[ngerman]{babel}
\usepackage[a4paper,top=2.5cm,bottom=3cm,left=2.5cm,right=2cm]{geometry}
\usepackage[T1]{fontenc}
\usepackage{booktabs}
\usepackage{graphicx}
\usepackage{csquotes}
\usepackage{paralist}
\usepackage{multirow,bigdelim}
\usepackage{setspace}
%\usepackage[math]{iwona}
\usepackage{textcomp}
\usepackage{listings}
\usepackage{xcolor}
\usepackage{pdfpages}
\usepackage{graphicx}
\usepackage{caption}
\usepackage{subfig}
\usepackage{amsmath}
\usepackage{amssymb}
\usepackage[shortlabels]{enumitem}
\usepackage{wrapfig}
\usepackage[gen]{eurosym}
\usepackage{siunitx}
\usepackage{polynom}
\usepackage{tabularx}
\usepackage[weather]{ifsym}
%\usepackage{array}
\newcolumntype{A}{>{$}p{5cm}<{$}}
\usepackage{multicol}
\usepackage[
	colorlinks=true,
	urlcolor=blue,
	linkcolor=blue
]{hyperref}
\usepackage{siunitx}
\sisetup{
	locale = DE ,
	per-mode = symbol-or-fraction,
	fraction-function=\dfrac
}
\usepackage{MnSymbol,wasysym,pifont,units}

% \usepackage[pdf]{pstricks}
% \usepackage{pstricks-add}
% \usepackage{auto-pst-pdf}
\pointpoints{Punkt}{Punkte}
\bonuspointpoints{Bonuspunkt}{Bonuspunkte}
\renewcommand{\solutiontitle}{\noindent\textbf{Lösung:}\enspace}
 
\chqword{Frage}   
\chpgword{Seite} 
\chpword{Punkte}   
\chbpword{Bonus Punkte} 
\chsword{Erreicht}   
\chtword{Gesamt}
 
\pagestyle{headandfoot}
\runningheadrule
 
%%%%%%%
\definecolor{hellgelb}{rgb}{1,1,0.8}
\definecolor{lightgelb}{rgb}{1,1,0.8}
\definecolor{colKeys}{rgb}{0,0,1}
\definecolor{colIdentifier}{rgb}{0,0,0}
\definecolor{colComments}{rgb}{1,0,0}
\definecolor{colString}{rgb}{0,0.5,0}
 
\usepackage{listings}
\lstset{%
    float=hbp,%
    basicstyle=\ttfamily\footnotesize, %
    identifierstyle=\color{colIdentifier}, %
    keywordstyle=\color{colKeys}, %
    stringstyle=\color{colString}, %
    commentstyle=\color{colComments}, %
    columns=flexible, %
    tabsize=2, %
    frame=single, %
    upquote=true,%
    extendedchars=true, %
    showspaces=false, %
    showstringspaces=false, %
    numbers=left, %
    numberstyle=\tiny, %
    breaklines=true, %
    backgroundcolor=\color{hellgelb}, %
    breakautoindent=true, %
    captionpos=b%
}
 
%%%%%%%%%%%%
\lstset{literate=%
    {Ö}{{\"O}}1
    {Ä}{{\"A}}1
    {Ü}{{\"U}}1
    {ß}{{\ss}}1
    {ü}{{\"u}}1
    {ä}{{\"a}}1
    {ö}{{\"o}}1
    {~}{{\textasciitilde}}1
}
 
\usepackage{pgfpages}                                 % <— load the package
\usepackage{atbegshi}
 
\newcommand{\twoonone}{% 
  \pgfpagesuselayout{2 on 1}[a4paper,landscape,border shrink=5mm] % <— set options
  % duplicate the content at shipout time
  \AtBeginShipout{%
    \pgfpagesshipoutlogicalpage{1}\copy\AtBeginShipoutBox%
    \pgfpagesshipoutlogicalpage{2}\box\AtBeginShipoutBox%
    \pgfshipoutphysicalpage%
  }}

\setlength{\parindent}{0pt}
\setlength{\parskip}{6pt}
 
\usepackage{array}
\newcommand{\PreserveBackslash}[1]{\let\temp=\\#1\let\\=\temp}
\newcolumntype{C}[1]{>{\PreserveBackslash\centering}p{#1}}
\newcolumntype{R}[1]{>{\PreserveBackslash\raggedleft}p{#1}}
\newcolumntype{L}[1]{>{\PreserveBackslash\raggedright}p{#1}}
\usepackage{stackengine}
\newcommand\xrowht[2][0]{\addstackgap[.5\dimexpr#2\relax]{\vphantom{#1}}}


%\twoonone % two pages on one 

\firstpageheader{Mathematik 11h (Wissel)\\\today}{Wiederholung des Themas Funktionen}{\includegraphics[scale=0.38]{../logo.jpg}}
\runningheader{Mathematik 11h (Wissel)\\\today}{Wiederholung des Themas Funktionen}{\includegraphics[scale=0.38]{../logo.jpg}}
\firstpagefooter{}{}{\thepage\,/\,\numpages}
\runningfooter{}{}{\thepage\,/\,\numpages}

\begin{document}
\vspace*{0.3cm}
\begin{center}
	\huge\bfseries Handout 04: Lineare und Quadratische Funktionen
\end{center}

\section*{Hausaufgabe}

\par Neben den Lösungen der Aufgaben A, B und C aus dem Arbeitsblatt zur Rekonstruktion und Modellierung, löst auch gemeinsam die Aufgaben des Stationenlernens und einzeln die Aufgaben auf diesem Handout. Bitte bearbeitet minimal drei! Unteraufgaben pro Aufgabe.

\par Wie immer optional könnt ihr die (handschriftlichen) Ausarbeitungen zu den folgenden Aufgaben dieses Handouts auch in digitaler Form (pdf) bis zum 24.9. auf Nextcloud oder LANIS hochladen. Achtet dabei, die Dateien sinnvoll (ohne Umlaute) und mit einem Bezug zum Handout zu benennen.\\ \textbf{Wichtig:} Vergesst nicht, den Haken in LANIS zu setzen, wenn ihr die Hausaufgabe bearbeitet habt.

\section*{Übungen}

\begin{questions}
	\printanswers
	\question Berechne die Funktionsgleichung der linearen Funktion, die die Punkte A und B enthält.
	\begin{align*}
		\text{a) } & A=(1|3),\; B=(5|11)   & \text{b) }          & A=(43|-2),\; B=(1|-2) & \text{c) }          & A=(5|23),\; B=(5|1) \\
		\text{d) } & A=(0|-1),\; B=(-2|-4)
		           & \text{e) }            & A=(0|0),\; B=(-1|1) & \text{f) }            & A=(4|8),\; B=(3|-7)
	\end{align*}

	\emph{mögliche Lösungen:} $f(x)=-2$; $x=5$ (\emph{Kommentar wichtig!});
	$f(x)=2x+1$; $f(x)=-x$; $f(x)=1,5x-1$; $f(x)=15x-52$

	\begin{solution} Zweipunkteform der Geradengleichung\\
		Beispiel: a) \begin{align*}
			f(x) & =\dfrac{f(x_A)-f(x_B)}{x_A-x_B}\cdot (x-x_A)+f(x_A) \\
			     & =\dfrac{3-11}{1-5}\cdot (x-1)+3=2(x-1)+3=2x+1\end{align*}
		\begin{align*}
			\text{a) } & f(x)=2x+1   & \text{b) } & f(x)=-2    & \text{c) }  & x=5 (\text{keine! Funktion}) \\
			\text{d) } & f(x)=1,5x-1
			           & \text{e) }  & f(x)=-x    & \text{f) } & f(x)=15x-52
		\end{align*}
	\end{solution}
	\question Berechne den Steigungswinkel der folgenden Funktionen:
	\begin{align*}
		\text{a) } & f(x)=5x+2    & \text{b) } & f(x)=-2x+4 & \text{c) } & f(x)=100x+10 \\
		\text{d) } & f(x)=-0,5x-5
		           & \text{e) }   & f(x)=90x   & \text{f) } & f(x)=10
	\end{align*}
	\emph{mögliche Lösungen (nur gerundete Ergebnisse):} $116,57^\circ$; $153,43^\circ$;
	$89,36^\circ$; $0^\circ$; $78,69^\circ$; $89,43^\circ$
	\begin{solution} Beispiel: $\alpha=\arctan(m)$
		\begin{align*}
			\text{a) } & \alpha=78,69^\circ                  & \text{b) }  & \alpha=-63,43^\circ+180^\circ=116,57^\circ & \text{c) } & 89,43^\circ \\
			\text{d) } & -26,57^\circ+180^\circ=153,43^\circ
			           & \text{e) }                          & 89,36^\circ & \text{f) }                                 & 0^\circ
		\end{align*}
	\end{solution}

	\question Weise nach, dass der gegebene Punkt auf dem Graphen der Funktion $f(x)=-3x+8$ liegt, bzw. gib an, welchen Wert der enthaltene Parameter annehmen muss, damit der Punkt auf dem Graphen liegt:
	\begin{align*}
		\text{a) } & P=(1|5) & \text{b) } & P=(-1,5|12,5) & \text{c) } & P=(a|100)
	\end{align*}
	\emph{zur Lösung:} Hier muss entweder durch logische Argumentation oder eine Rechnung gezeigt werden, dass die Punkte auf der Geraden liegen. Es reicht nicht zu sagen: „Der Punkt liegt drauf, weil ich das sehe...``
	oder ähnliches. Legt eure Lösung jemand anderem vor und dieser muss eure Schritte ohne zusätzliche Erklärung sofort und eindeutig nachvollziehen können.
	\begin{solution}
		\begin{align*}
			 & \text{a) }  f(1)=-3\cdot 1+ 8 = 5 \quad \Rightarrow \text{Der Punkt P liegt auf der dem Graphen}                     \\
			 & \text{b) }  f(-1,5)=-3\cdot -1,5 + 8 = 12,5 \quad \Rightarrow \text{Der Punkt P liegt ebenfalls auf der dem Graphen} \\
			 & \text{c) }  f(a)=-3\cdot a+ 8 = 100 \quad \Rightarrow a=-\frac{92}{3}                                                \\
			 & \phantom{c) }  \quad \Rightarrow \text{Der Punkt P liegt auf der dem Graphen wenn $a=-\frac{92}{3}$ ist.}
		\end{align*}
	\end{solution}

	\question Bestimme die relative Lage der beiden Geraden und berechne gegebenenfalls den Schnittpunkt und den Schnittwinkel.
	\begin{align*}
		\text{a) } & f(x)=2x+4,\; g(x)=3x-5        & \text{b) }                   & f(x)=3x+2x-8+x,\; g(x)=6x+8 \\
		\text{c) } & f(x)=5x-2+3+x,\; g(x)=3x+3x+1
		           & \text{d) }                    & f(x)=2x-6+x-1,\; g(x)=-4-3-x
	\end{align*}
	\emph{mögliche Lösungen:} $S=(0|-7)$;\; $S=(9|22)$, parallel (\emph{Begründung!}); identisch (\emph{Begründung!})
	\begin{solution}\begin{itemize}
			\item[a) ] Schnittpunkt:
			      \begin{align*}
				      f(x)=g(x) \Leftrightarrow 2x+4=3x-5 \Leftrightarrow x=9 \Rightarrow S(9|f(9)=22) \text{ ist ein Schnittpunkt}
			      \end{align*}
			      Schnittwinkel:
			      \begin{align*}
				       & \alpha_f=\arctan(2)=63,43^\circ                                                        \\
				       & \alpha_g=\arctan(3)=71,56^\circ                                                        \\
				       & \varphi=\left|\alpha_f-\alpha_g\right|=\left|63,43^\circ-71,56^\circ\right|=8,13^\circ
			      \end{align*}
			\item[b) ] Mit $f(x)=3x+2x-8+x=6x-8$ hat $f(x)$ die gleiche Steigung wie $g(x)$ hat, sind beide Geraden parallel.
			\item[c) ] Mit $f(x)=5x-2+3+x=6x+1$ und $g(x)=3x+3x+1=6x+1$ sind beide Geraden identisch, da sie die gleiche Funktionsgleichung besitzen.
			\item[d) ] Mit $f(x)=2x-6+x-1=3x-7$ und $g(x)=-4-3-x=-x-7$ ist der Schnittpunkt:
			      \begin{align*}
				      f(x)=g(x) \Leftrightarrow 3x-7=-x-7 \Leftrightarrow x=0 \Rightarrow S(0|f(0)=-7).
			      \end{align*}
			      Der Schnittwinkel entsprechend:
			      \begin{align*}
				       & \alpha_f=\arctan(3)=71,56^\circ                                                        \\
				       & \arctan(-1)=-45^\circ \Rightarrow \alpha_g=180^\circ-45^\circ=135^\circ                \\
				       & \varphi=\left|\alpha_f-\alpha_g\right|=\left|71,56^\circ-135^\circ \right|=63,44^\circ
			      \end{align*}
		\end{itemize}

	\end{solution}

	% \question Bestimme die Gleichung der orthogonalen Geraden zum Graphen
	% von f, die durch P=(5|4) geht.

	% a) f(x)=10x+1b) f(x)=-0,5x+8

	% \emph{mögliche Lösungen:} f\textsubscript{ort}(x)=-0,1x+4,5;
	% f\textsubscript{ort}(x)=2x-6

	% \question Bestimme den Scheitelpunkt und die Nullstellen der
	% quadratischen Funktion.

	% a) f(x)=-2x²+4x+8b) f(x)=-x²+10x-25c) f(x)=3x²+x+2

	% \emph{mögliche Lösungen für Scheitelpunkte:} S=(5|0); S=(-1/6
	% |{} 23/12); S=(1|10)

	% \emph{mögliche Lösungen für Nullstellen (gerundet):} N=(3,24|0);
	% N=(5|0); N=(-1,24|0); keine NS

	% \question Bestimme die Intervalle, in denen der Parameter a liegen
	% muss, damit die Funktion keine, eine oder zwei Nullstellen hat.

	% a) f(x)=x²+4x+a b) f(x)=x²+2ax+9

	% mögliche Lösungen:

	% 1)\textbf{ }keine → -3\textless a\textless3, eine → a=3 od. a=-3, zwei →
	% a\textless-3 und a\textgreater3 (\emph{Herleitung wichtig!!!})

	% 2) keine → a\textgreater4, eine → a=4, zwei → a\textless4
	% (\emph{Herleitung wichtig!!!})

	% \question Bestimme die relative Lage der Graphen von f und g und gib
	% gegebenenfalls die Schnittpunkte an.

	% a) f(x)=4x²+x-1, g(x)=x+1b) f(x)=x²+5x+8, g(x)=0,25x-9

	% \emph{mögliche Lösungen (gerundet):} Passante; Sekante mit S1=(0,71
	% |{} 1,71) und S2=(-0,71 |{} 0,29)

\end{questions}

\par \textbf{Tipp}: Die zu dem Thema zugehörige Playlist von Daniel Jung lautet \href{https://t1p.de/ix37}{Quadratische Funktionen, Parabeln\footnote{\url{https://t1p.de/ix37}}} und \href{https://t1p.de/ni5e}{Lineare Funktionen (Geraden), y=m*x+n\footnote{\url{https://t1p.de/ni5e}}}, siehe auch Lesezeichen auf Nextcloud.
\vspace{1cm}

% \begin{wrapfigure}{H!}{5cm} 
\includegraphics[scale=0.4]{qr-code-t1p-de-a6i1}\\
%  \text{Feedback: \hyperlink{https://t1p.de/9c50}{https://t1p.de/9c50}}
Feedback: \href{https://t1p.de/a6i1}{https://t1p.de/a6i1}

\end{document}
