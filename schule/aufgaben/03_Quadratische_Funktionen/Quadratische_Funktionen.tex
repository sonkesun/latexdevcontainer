%% !TEX TS-program = Arara
%% arara: pdflatex: {shell: yes}
%% arara: pdflatex: {shell: yes}
%% arara: clean: { extensions: [ log, aux, nav, out, snm, vrb, toc ] }
 
\documentclass[a4paper,ngerman,12pt]{exam} 


\usepackage[ngerman]{babel}
\usepackage[a4paper,top=2.5cm,bottom=3cm,left=2.5cm,right=2cm]{geometry}
\usepackage[T1]{fontenc}
\usepackage{booktabs}
\usepackage{graphicx}
\usepackage{csquotes}
\usepackage{paralist}
\usepackage{multirow,bigdelim}
\usepackage{setspace}
%\usepackage[math]{iwona}
\usepackage{textcomp}
\usepackage{listings}
\usepackage{xcolor}
\usepackage{pdfpages}
\usepackage{graphicx}
\usepackage{caption}
\usepackage{subfig}
\usepackage{amsmath}
\usepackage{amssymb}
\usepackage[shortlabels]{enumitem}
\usepackage{wrapfig}
\usepackage[gen]{eurosym}
\usepackage{siunitx}
\usepackage{polynom}
\usepackage{tabularx}
\usepackage[weather]{ifsym}
%\usepackage{array}
\newcolumntype{A}{>{$}p{5cm}<{$}}
\usepackage{multicol}
\usepackage[
	colorlinks=true,
	urlcolor=blue,
	linkcolor=blue
]{hyperref}
\usepackage{siunitx}
\sisetup{
	locale = DE ,
	per-mode = symbol-or-fraction,
	fraction-function=\dfrac
}
\usepackage{MnSymbol,wasysym,pifont,units}

% \usepackage[pdf]{pstricks}
% \usepackage{pstricks-add}
% \usepackage{auto-pst-pdf}

\pointpoints{Punkt}{Punkte}
\bonuspointpoints{Bonuspunkt}{Bonuspunkte}
\renewcommand{\solutiontitle}{\noindent\textbf{Lösung:}\enspace}
 
\chqword{Frage}   
\chpgword{Seite} 
\chpword{Punkte}   
\chbpword{Bonus Punkte} 
\chsword{Erreicht}   
\chtword{Gesamt}
 
\pagestyle{headandfoot}
\runningheadrule
 
%%%%%%%
\definecolor{hellgelb}{rgb}{1,1,0.8}
\definecolor{lightgelb}{rgb}{1,1,0.8}
\definecolor{colKeys}{rgb}{0,0,1}
\definecolor{colIdentifier}{rgb}{0,0,0}
\definecolor{colComments}{rgb}{1,0,0}
\definecolor{colString}{rgb}{0,0.5,0}
 
\usepackage{listings}
\lstset{%
    float=hbp,%
    basicstyle=\ttfamily\footnotesize, %
    identifierstyle=\color{colIdentifier}, %
    keywordstyle=\color{colKeys}, %
    stringstyle=\color{colString}, %
    commentstyle=\color{colComments}, %
    columns=flexible, %
    tabsize=2, %
    frame=single, %
    upquote=true,%
    extendedchars=true, %
    showspaces=false, %
    showstringspaces=false, %
    numbers=left, %
    numberstyle=\tiny, %
    breaklines=true, %
    backgroundcolor=\color{hellgelb}, %
    breakautoindent=true, %
    captionpos=b%
}
 
%%%%%%%%%%%%
\lstset{literate=%
    {Ö}{{\"O}}1
    {Ä}{{\"A}}1
    {Ü}{{\"U}}1
    {ß}{{\ss}}1
    {ü}{{\"u}}1
    {ä}{{\"a}}1
    {ö}{{\"o}}1
    {~}{{\textasciitilde}}1
}
 
\usepackage{pgfpages}                                 % <— load the package
\usepackage{atbegshi}
 
\newcommand{\twoonone}{% 
  \pgfpagesuselayout{2 on 1}[a4paper,landscape,border shrink=5mm] % <— set options
  % duplicate the content at shipout time
  \AtBeginShipout{%
    \pgfpagesshipoutlogicalpage{1}\copy\AtBeginShipoutBox%
    \pgfpagesshipoutlogicalpage{2}\box\AtBeginShipoutBox%
    \pgfshipoutphysicalpage%
  }}

\setlength{\parindent}{0pt}
\setlength{\parskip}{6pt}
 
\usepackage{array}
\newcommand{\PreserveBackslash}[1]{\let\temp=\\#1\let\\=\temp}
\newcolumntype{C}[1]{>{\PreserveBackslash\centering}p{#1}}
\newcolumntype{R}[1]{>{\PreserveBackslash\raggedleft}p{#1}}
\newcolumntype{L}[1]{>{\PreserveBackslash\raggedright}p{#1}}
\usepackage{stackengine}
\newcommand\xrowht[2][0]{\addstackgap[.5\dimexpr#2\relax]{\vphantom{#1}}}


%\twoonone % two pages on one 

\firstpageheader{Mathematik 11h (Wissel)\\\today}{Wiederholung des Themas Funktionen}{\includegraphics[scale=0.38]{../logo.jpg}}
\runningheader{Mathematik 11h (Wissel)\\\today}{Wiederholung des Themas Funktionen}{\includegraphics[scale=0.38]{../logo.jpg}}
\firstpagefooter{}{}{\thepage\,/\,\numpages}
\runningfooter{}{}{\thepage\,/\,\numpages}


\begin{document}

\vspace*{0.3cm}
\begin{center}
  \huge\bfseries Handout 03:\\ Quadratische Funktionen\\ (Die Normalparabel)
\end{center}

\section*{Hausaufgabe}

\par Bitte lest zur nächsten Präsenzstunde die Seiten $32$ bis $35$ im Buch und bearbeitet damit die Übungen dieses Handouts bzw. aus dem Buch.

\par Wie immer optional könnt ihr die (handschriftlichen) Ausarbeitungen zu den folgenden Aufgaben dieses Handouts auch in digitaler Form (pdf) bis zum 10.9. auf Nextcloud oder LANIS hochladen. Achtet dabei, die Dateien sinnvoll (ohne Umlaute) und mit einem Bezug zum Handout zu benennen.\\ \textbf{Wichtig:} Vergesst nicht, den Haken in LANIS zu setzen, wenn ihr die Hausaufgabe bearbeitet habt.

\section*{Übungen}

\begin{questions}
  \printanswers
  \question Übung 3 im Buch auf Seite 33
  \begin{solution}
    \begin{itemize}
      \item[a) ] $f(x)=x^2+7$
      \item[b) ] $f(x)=x^2-3$
      \item[c) ] $f(x)=x^2-20$
      \item[d) ] $f(x)=x^2+4$
    \end{itemize}
  \end{solution}
  \question Übung 5 im Buch auf Seite 34
  \begin{solution} Eine Verschiebung der Normalparabel entlang der $x$-Achse, muss zu einer Funktion der Form $f(x)=(x-a)^2$ führen. Daraus folgt:
    \begin{itemize}
      \item[a) ] nein: $g(x)=x^2+x+1=x^2+x+\left(\frac{1}{2}\right)^2-\left(\frac{1}{2}\right)^2+1= (x-\frac{1}{2})^2+...$
      \item[b) ] ja: $ g(x)=x^2+2x+1=x^2+2x+1^2-1+1= (x+1)^2$
      \item[c) ] nein: $g(x)=x^2-6x=(x-0)^2-...$
      \item[d) ] nein: $g(x)=2(x-0)^2-...$
    \end{itemize}
  \end{solution}
  \question Übung 8 im Buch auf Seite 35
  \begin{solution} Ansatz ($x$-Achse entlang der Fahrbahnebene): $y(x)=a(x-0)^2+7$ mit $a\in\mathbb{R}$\\[3ex]
    Mit $y(20)=17$ folgt: $a(20)^2+7=17$
    \begin{itemize}
      \item[a) ]  $\Rightarrow \quad a=\dfrac{1}{40}$; $y=\dfrac{1}{40}x^2+7$
      \item[b) ] und somit $l_1=y(20)=\SI{17}{m}$; $l_2=y(15)=\SI{12,625}{m}$; $l_3=y(10)=\SI{9,5}{m}$; $l_4=y(5)=\SI{7,625}{m}$
    \end{itemize}
  \end{solution}
  %################################################################################
  \question %12
  Es werden je drei Transformationen nacheinander ausgeführt. Ergänzt die Funktionsgleichungen und vergleicht diese in Geogebra\\[2ex]
  Abkürzungen der Transformationen:\\
  Vx: Verschieben um $2$ nach links\\
  St: Strecken um den Faktor $2$\\
  Vy: Verschieben um 3 nach unten\\
  Sx: Spiegeln an der x-Achse\\
  Sy: Spiegeln an der y-Achse\\
  \begin{table}[h!]
    %    \setlength{\tabcolsep}{10pt}
    \begin{tabular}{|m{3cm}|m{4cm}|m{4cm}|m{4cm}|l}
      \cline{1-4}
      \textbf{Funktion am Anfang} & \textbf{Ergebnis der ersten Veränderung} & \textbf{Ergebnis der zweiten Veränderung} & \textbf{Ergebnis der dritten Veränderung} & \\[4ex] \cline{1-4}
      $f_0(x)=3x^2$               & Vx: $f_1(x)=$                            & Sy: $f_2(x)=3(x-2)^2$                     & Vy: $f_3(x)=$                             & \\[3ex] \cline{1-4}
      $g_0(x)=(x-2)^2$            & Sx: $g_1(x)=$                            & Vy: $g_2(x)=$                             & St: $g_3(x)=$                             & \\[3ex] \cline{1-4}
      $h_0(x)=2x^2+1$             & Sy: $h_1(x)=$                            & Sx: $h_2(x)=$                             & Vx: $h_3(x)=$                             & \\[3ex] \cline{1-4}
    \end{tabular}
  \end{table}
  %################################################################################
  \begin{solution}\phantom{x}\\
    \begin{small}
      %    \begin{table}[]
      %    \setlength{\tabcolsep}{10pt}
      \begin{tabular}{|m{2.5cm}|m{3.5cm}|m{3.5cm}|m{4cm}|l}
        \cline{1-4}
        \textbf{Funktion am Anfang} & \textbf{Ergebnis der ersten Veränderung} & \textbf{Ergebnis der zweiten Veränderung} & \textbf{Ergebnis der dritten Veränderung} & \\[4ex] \cline{1-4}
        $f_0(x)=3x^2$               & $f_1(x)=3(x+2)^2$                        & $f_2(x)=3(x-2)^2$                         & $f_3(x)=3(x-2)^2-3$                       & \\[3ex] \cline{1-4}
        $g_0(x)=(x-2)^2$            & $g_1(x)=-(x-2)^2$                        & $g_2(x)=-(x-2)^2-3$                       & $g_3(x)=-2(x-2)^2-3$                      & \\[3ex] \cline{1-4}
        $h_0(x)=2x^2+1$             & $h_1(x)=2x^2+1$                          & $h_2(x)=-2x^2-1$                          & $h_3(x)=-2(x+2)^2-1$                      & \\[3ex] \cline{1-4}
      \end{tabular}
      %   \end{table}
    \end{small}
  \end{solution}

\end{questions}

\par \textbf{Tipp:} Die zu dem Thema zugehörige Playlist von Daniel Jung lautet \href{https://t1p.de/ix37}{Quadratische Funktionen, Parabeln\footnote{\url{https://t1p.de/ix37}}}, siehe auch Lesezeichen auf Nextcloud.

% \begin{wrapfigure}{H!}{5cm} 
\includegraphics[scale=0.4]{qr-code-t1p-de-xdox}\\
%  \text{Feedback: \hyperlink{https://t1p.de/9c50}{https://t1p.de/9c50}}
Feedback: \href{https://t1p.de/xdox}{https://t1p.de/xdox}

%\newpage
%
\end{document}